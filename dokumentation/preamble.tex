% ##################################################
% Unterstuetzung fuer die deutsche Sprache
% ##################################################
\usepackage{ngerman}
\usepackage[ngerman]{babel}

% ##################################################
% Dokumentvariablen
% ##################################################

% Persoenliche Daten
\newcommand{\docNameA}{Marcel}
\newcommand{\docNameB}{Manu}
\newcommand{\docNameC}{Matthieu}
\newcommand{\docNameD}{Martin}
\newcommand{\docNameE}{Kai}

\newcommand{\kaspar}{~Prof. Dr. Kaspar~}

% Dokumentdaten
\newcommand{\docTitle}{Semesterprojekt 2016}
%\newcommand{\docUntertitle}{} % Kein Untertitel
\newcommand{\docUntertitle}{Verborgene Bugs in C-Code einbauen und finden}
%Studiengaenge: Allgemeine Informatik Bachelor, Computer Networking Bachelor,
% Software-Produktmanagement Bachelor, Advanced Computer Scinece Master
\newcommand{\docErsterReferent}{\kaspar}
%\newcommand{\docZweiterReferent}{-} % Wenn es nur einen Betreuer gibt
\newcommand{\docZweiterReferent}{-}

% ##################################################
% Allgemeine Pakete
% ##################################################

% Abbildungen einbinden
\usepackage{graphicx}

% Zusaetsliche Sonderzeichen
\usepackage{dingbat}

% Farben
\usepackage{color}
\usepackage[usenames,dvipsnames,svgnames,table]{xcolor}

% Maskierung von URLs und Dateipfaden
\usepackage[hyphens]{url}

% Deutsche Anfuehrungszeichen
\usepackage[babel, german=quotes]{csquotes}

% Pakte zur Index-Erstellung (Schlagwortverzeichnis)
\usepackage{index}
\makeindex

% Ipsum Lorem
% Paket wird nur für das Beispiel gebraucht und kann gelöscht werden
\usepackage{lipsum}

% ##################################################
% Seitenformatierung
% ##################################################
\usepackage[
	portrait,
	bindingoffset=1.5cm,
	inner=2.5cm,
	outer=2.5cm,
	top=3cm,
	bottom=2cm,
	%includeheadfoot
	]{geometry}

% ##################################################
% Kopf- und Fusszeile
% ##################################################

\usepackage{fancyhdr}

\pagestyle{fancy}
\fancyhf{}

\fancyhead[ER,OR]{\sffamily SS16}
\fancyhead[EL,OL]{\sffamily Semesterprojekt}
\fancyfoot[EL,OL]{\sffamily Gruppe AC9: Manuel, Marcel, Matthieu, Martin, Kai}
\fancyfoot[ER,OR]{\sffamily \thepage}

\fancypagestyle{headings}{}

\fancypagestyle{plain}{}

\fancypagestyle{empty}{
  \fancyhf{}
  \renewcommand{\headrulewidth}{0pt}
}

%Kein "Kapitel # NAME" in der Kopfzeile
\renewcommand{\chaptermark}[1]{
	\markboth{#1}{}
   	\markboth{\thechapter.\ #1}{}
}

% ##################################################
% Schriften
% ##################################################

% Stdandardschrift festlegen
\renewcommand{\familydefault}{\sfdefault}

% Standard Zeilenabstand: 1,5 zeilig
\usepackage{setspace}
\onehalfspacing

% Schriftgroessen festlegen
\addtokomafont{chapter}{\sffamily\large\bfseries}
\addtokomafont{section}{\sffamily\normalsize\bfseries}
\addtokomafont{subsection}{\sffamily\normalsize\mdseries}
\addtokomafont{caption}{\sffamily\normalsize\mdseries}

% ##################################################
% Inhaltsverzeichnis / Allgemeine Verzeichniseinstellungen
% ##################################################

\usepackage{tocloft}

% Punkte auch bei Kapiteln
\renewcommand{\cftchapdotsep}{3}
\renewcommand{\cftdotsep}{3}

% Schriftart und -groesse im Inhaltsverzeichnis anpassen
\renewcommand{\cftchapfont}{\sffamily\normalsize}
\renewcommand{\cftsecfont}{\sffamily\normalsize}
\renewcommand{\cftsubsecfont}{\sffamily\normalsize}
\renewcommand{\cftchappagefont}{\sffamily\normalsize}
\renewcommand{\cftsecpagefont}{\sffamily\normalsize}
\renewcommand{\cftsubsecpagefont}{\sffamily\normalsize}

%Zeilenabstand in den Verzeichnissen einstellen
\setlength{\cftparskip}{.5\baselineskip}
\setlength{\cftbeforechapskip}{.1\baselineskip}

% ##################################################
% Abbildungsverzeichnis und Abbildungen
% ##################################################

\usepackage{caption}

\usepackage{wrapfig}

% Nummerierung von Abbildungen
\renewcommand{\thefigure}{\arabic{figure}}
\usepackage{chngcntr}
\counterwithout{figure}{chapter}

% Abbildungsverzeichnis anpassen
\renewcommand{\cftfigpresnum}{Abbildung }
\renewcommand{\cftfigaftersnum}{:}

% Breite des Nummerierungsbereiches [Abbildung 1:]
\newlength{\figureLength}
\settowidth{\figureLength}{\bfseries\cftfigpresnum\cftfigaftersnum}
\setlength{\cftfignumwidth}{\figureLength}
\setlength{\cftfigindent}{0cm}

% Schriftart anpassen
\renewcommand\cftfigfont{\sffamily}
\renewcommand\cftfigpagefont{\sffamily}

% ##################################################
% Tabellenverzeichnis und Tabellen
% ##################################################

% Nummerierung von Tabellen
\renewcommand{\thetable}{\arabic{table}}
\counterwithout{table}{chapter}

% Tabellenverzeichnis anpassen
\renewcommand{\cfttabpresnum}{Tabelle }
\renewcommand{\cfttabaftersnum}{:}

% Breite des Nummerierungsbereiches [Abbildung 1:]
\newlength{\tableLength}
\settowidth{\tableLength}{\bfseries\cfttabpresnum\cfttabaftersnum}
\setlength{\cfttabnumwidth}{\tableLength}
\setlength{\cfttabindent}{0cm}

%Schriftart anpassen
\renewcommand\cfttabfont{\sffamily}
\renewcommand\cfttabpagefont{\sffamily}

% Unterdrueckung von vertikalen Linien
\usepackage{booktabs}

% ##################################################
% Listings (Quellcode)
% ##################################################

\usepackage{listings}
\lstset{
	language=c,
	backgroundcolor=\color{white},
	breaklines=true,
	prebreak={\carriagereturn},
 	breakautoindent=true,
 	numbers=left,
 	numberstyle=\tiny,
 	stepnumber=2,
 	numbersep=5pt,
 	keywordstyle=\color{blue},
   	commentstyle=\color{green},
   	stringstyle=\color{gray}
}

% ##################################################
% Theoreme
% ##################################################

% Umgebung fuer Beispiele
\newtheorem{beispiel}{Beispiel}

% Umgebung fuer These
\newtheorem{these}{These}

% Umgebung fuer Definitionen
\newtheorem{definition}{Definition}

% ##################################################
% Literaturverzeichnis
% ##################################################

\usepackage{bibgerm}

% ##################################################
% Abkuerzungsverzeichnis
% ##################################################

\usepackage[printonlyused]{acronym}

% ##################################################
% PDF / Dokumenteninternelinks
% ##################################################

\usepackage[
    colorlinks=false,
    linkcolor=black,
    citecolor=black,
    filecolor=black,
    urlcolor=black,
    bookmarks=true,
    bookmarksopen=true,
    bookmarksopenlevel=3,
    bookmarksnumbered,
    plainpages=false,
    pdfpagelabels=true,
    hyperfootnotes,
    pdftitle ={\docTitle}]{hyperref}

\providecommand{\tightlist}{%
  \setlength{\itemsep}{0pt}\setlength{\parskip}{0pt}}
