\chapter{Einleitung}

\section{Projektumfeld}\label{projektumfeld}

Das Projekt wird von Studenten der HFU als Semesterprojekt durchgeführt.
Die Idee kam von \kaspar der dieses Projekt im Sommersemester 2016
bereit stellte. Das Projekt wurde folgendermassen beworben:

\begin{quote}
In diesem Projekt soll gezielt verborgene Funktionalität vorzugsweise
als Bugs in C oder C++ Code eingebaut werden. Jeder Teilnehmer überlegt
sich, wie er möglichst schwer zu erkennende Bugs, die als verborgene
Hintertür genutzt werden können, in eine Software einbaut. In einer
zweiten Rolle sucht jeder Teilnehmer Bugs bzw. Hintertüren in der
Software anderer Projektteilnehmer. In einem ersten Schritt steht dafür
nur der Binärcode zur Verfügung, im zweiten Schritt dann zusätzlich der
Quelltext. Werkzeuge erleichtern das Auffinden von Bugs bzw.
Hintertüren. Es gilt dabei möglichst produktiv zu sein, das bedeutet,
die richtigen Werkzeuge zu verwenden.
\end{quote}

Nach dem erstem Treffen hat sich dann diese Vorgehensweise ergeben:

\begin{enumerate}
\def\labelenumi{\arabic{enumi}.}
\tightlist
\item
  Zwei Teams werden gebildet

  \begin{itemize}
  \tightlist
  \item
    Ein Team entwickelt eine Executable für die andere
  \end{itemize}
\item
  Nun wird versucht die Hintertür zu finden und auszunutzen.
\item
  Als letzer Schritt wird der Source Code auch preisgegeben
\item
  Es sollte am besten in beiden Versionen nicht sichtbar sein
\item
  Nach etwa 4 Wochen werden die Teams neu zusammengestellt und der
  Prozess wiederholt.
\end{enumerate}

\section{Projektverlauf}\label{projektverlauf}

Der Projektverlauf verlief nur mit wenigen Organisationsproblemen und
lief somit nach dem oben angegebenem Plan. Das Endresultat besteht aus
vier Executables in C geschrieben die verschiedene Schwachstellen
darstellen:

\begin{itemize}
\tightlist
\item
  Eine Bank Emulation die Signale benutzt um eine Hintertür zu öffnen.
\item
  Eine Kalendar Applikation mit einem Timing um hereinzukommen
\item
  Ein Remote Server mit eingebautem Fehlerverhalten
\item
  Eine gepatchte \texttt{su} executable mit einem Bufferoveflow um root
  rechte zu bekommen.
\end{itemize}

Diese wurden auf ein Online Repository\footnote{https://github.com/asuivelentine/bughack}
hochgeladen und sind dort ersichtlich.
